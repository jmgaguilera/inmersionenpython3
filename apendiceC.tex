% apendiceC.tex
% This work is licensed under the Creative Commons Attribution-Noncommercial-Share Alike 3.0 License.
% To view a copy of this license, visit http://creativecommons.org/licenses/by-nc-sa/3.0/nz
% or send a letter to Creative Commons, 171 Second Street, Suite 300, San Francisco, California, 94105, USA.

\chapter{Dónde continuar}



\begin{citaCap}
    ``Avanza en tu camino, ya que solo existe mientras estás caminando.'' \\
        ---\emph{San Agustín de Hipona (atribuido)}
\end{citaCap}

\section{Cosas para leer}

Desafortunadamente, no puede cubrir todos los aspectos de Python 3 en este libro. Afortunadamente, hay muchos tutoriales maravillosos libremente disponibles.

Sobre decoradores:

\begin{itemize}
  \item \href{http://programmingbits.pythonblogs.com/27\_programmingbits/archive/50\_function\_decorators.html}{Decoradores de función}, por Ariel Ortiz.
  \item \href{http://programmingbits.pythonblogs.com/27\_programmingbits/archive/51\_more\_on\_function\_decorators.html}{Más sobre decoradores}, por Ariel Ortiz.
  \item \href{http://www.ibm.com/developerworks/linux/library/l-cpdecor.html}{Python encantador: los decoradores hacen magia fácilmente}, por David Mertz.
  \item \href{http://docs.python.org/reference/compound\_stmts.html\#function}{Definiciones de función} en la documentación oficial de Python.
\end{itemize}

Sobre propiedades:

\begin{itemize}
  \item \href{http://adam.gomaa.us/blog/2008/aug/11/the-python-property-builtin/}{La función de Python \codigo{property()}}, por Adam Gomaa.
  \item \href{http://tomayko.com/writings/getters-setters-fuxors}{Getters/Setters/Fuxors}, por Ryan Tomayko.
  \item \href{http://docs.python.org/library/functions.html#property}{La función \codigo{property()}} en la documentación oficial de Python.
\end{itemize}

Sobre descriptores:

\begin{itemize}
  \item \href{http://users.rcn.com/python/download/Descriptor.htm}{La guía sobre cómo usar los descriptores}, por Raymond Hettinger.
  \item \href{http://www.ibm.com/developerworks/linux/library/l-python-elegance-2.html}{Python encantador: elegancia y ``verrugas'', parte 2}, por David Mertz.
  \item \href{http://www.informit.com/articles/printerfriendly.aspx?p=1309289}{Descriptores en Python}, por Mark Summerfield.
  \item \href{http://docs.python.org/3.1/reference/datamodel.html#invoking-descriptors}{Invocando descriptores} en la documentación oficial de Python.
\end{itemize}

Sobre hilos y multiproceso:

\begin{itemize}
  \item \href{http://docs.python.org/3.1/library/threading.html}{el módulo \codigo{threading}}.
  \item \href{http://www.doughellmann.com/PyMOTW/threading/}{\codigo{threading} --- Gestionar hilos concurrentes}.
  \item \href{http://docs.python.org/3.1/library/multiprocessing.html}{El módulo \codigo{multiprocessing}}.
  \item \href{http://www.doughellmann.com/PyMOTW/multiprocessing/}{\codigo{multiprocessing} --- Gestionar procesos como hilos}.
  \item \href{http://jessenoller.com/2009/02/01/python-threads-and-the-global-interpreter-lock/}{Los hilos de Python y el bloque global de intérprete}, por Jesse Noller.
  \item \href{http://blip.tv/file/2232410}{Dentro de Python \codigo{GIL}(video)}, por David Beazley.
\end{itemize}

Sobre metaclases:

\begin{itemize}
  \item \href{http://gnosis.cx/publish/programming/metaclass_1.html}{Programación con metaclases en Python}, por David Mertz y Michele Simionato.
  \item \href{http://www.phyast.pitt.edu/~micheles/python/meta2.html}{Programación con metaclases en Python, parte 2}, por David Mertz y Michele Simionato.
  \item \href{http://www.ibm.com/developerworks/library/l-pymeta3/}{Programación con metaclases en Python, parte 3}, por David Mertz y Michele Simionato.
\end{itemize}

Además, \href{https://pymotw.com/2/contents.html}{el módulo Python de la semana}, de Doug Hellman, es una fantástica guía para muchos de los módulos de la librería estándar de Python.
